\documentclass{beamer}
\usepackage[utf8]{inputenc}
\usepackage{lipsum}
\usepackage[T1]{fontenc}
\usepackage{amsfonts}
\usepackage{amsmath}
\usepackage{amsthm}
\usepackage{hyperref}
\usepackage[croatian]{babel}
\usepackage{graphicx}
\usepackage{verbatim}
\theoremstyle{definition}

\title{Granice za Remzijeve brojeve i primene}
\date{15.1.2020}
\author{Mihailo Milenković, Dejan Gjer, Bojana Čakarević}

\newtheorem{definicija}{Definicija}[section]
\newtheorem{teorema}{Teorema}[section]
\newtheorem{posledica}{Posledica}[teorema]
\newtheorem{lema}[teorema]{Lema}
\newcommand{\dokaz}[1]{\begin{proof}[Dokaz]#1\end{proof}}

\begin{document}
	
\frame{\titlepage}

\frame{
	\frametitle{Struktura rada}
	\begin{itemize}
		\item Gornje granice za Remzijeve brojeve
		\item Donje granice za Remzijeve brojeve
		\item Primene Remzijeve teoreme
	\end{itemize}
}
\frame{
	\frametitle{Remzijevi brojevi}
	
	\begin{definicija}
		Za skup $X$ i prirodan broj $k>0$ definišemo
		\[
		[X]^k :=\{Y\subseteq X | \:|Y|=k\}.
		\]
		Pišemo
		\[
		n\rightarrow (l_1,\ldots,l_r)
		\]
		ako za svako bojenje $\chi:[\underline{n}]^2$ postoje $i\in \underline{r}$ i $T\subseteq \underline{n}$ sa $|T|=l_i$, takvi da je $[T]^2$ u odnosu na $\chi$ $i$-monohromatsko.	
	\end{definicija}

	\begin{definicija}
		Najmanji broj $n$ za koji važi
		\[
		n\rightarrow(l_1,l_2)
		\]
		naziva se Remzijev broj i označavamo ga sa $R(l_1,l_2)$.
	\end{definicija}
}
\frame{
	\frametitle{Gornje granice za Remzijeve brojeve}
	\begin{teorema}
		\[
		R(l_1,l_2) \leq R(l_2-1, l_1) + R(l_2, l_1-1)
		\]
	\end{teorema}
	\begin{teorema}
		Za $k \geq 2$ važi \newline
		
		$ R(l_1, l_2, ... , l_k) \leq 2 + \sum\limits_{i=1}^{k}(R(l_1, l_2 ... , l_{i-1}, l_i-1, l_{i+1}, ... , l_k)-1) $
	\end{teorema}
}
\frame{
	\frametitle{Donje granice za Remzijeve brojeve}
	\begin{teorema}\label{dot1}
		Neka su dati prirodni brojevi $n$ i $k$, takvi da $n \geq{k} > 0$ .Ako je $$\binom{n}{k}2^{1 - \binom{k}{2}} < 1 ,$$  onda važi $R(k, k) > n$.
	\end{teorema}
	\begin{teorema}\label{dot3}
		Neka su dati prirodni brojevi $n$, $m$ i $k$ tako da je $1\leq k \leq n - 2$. Tada je $$R(m,n) \geq R(m, n - k) + R(m, k + 1) - 1.$$
	\end{teorema}
}
\frame{
	\frametitle{Primene Remzijeve teoreme}
	\begin{teorema}\label{sur}
		Za svako $k \in \mathbb{N}\textbackslash \{0\}$  postoji neko $n_{0} \in \mathbb{N}$, takvo da za svako bojenje $\chi:\underline{n} \rightarrow \underline{k}$ postoje brojevi $x, y, z \in \underline{n}$ sa osobinom 
		\[
		x + y = z \: \mathrm{i} \: \chi(x)= \chi(y)=\chi(z)
		\]
	\end{teorema}
	\begin{teorema}
		Za svaki prirodan broj $n\geq 3$ postoji broj $N(n)$ takav da bilo koji skup od bar $N$ tačaka u ravni u opštem položaju sadrži konveksan $n$-tougao
	\end{teorema}
}
\end{document}