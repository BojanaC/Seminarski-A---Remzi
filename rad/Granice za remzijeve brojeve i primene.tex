\documentclass{article}
\usepackage{lipsum}
\usepackage[utf8]{inputenc}
\usepackage[T1]{fontenc}
\usepackage{amsfonts}
\usepackage{amsmath}
\usepackage{amsthm}
\usepackage{hyperref}
\title{Granice za Remzijeve brojeve i primene}
\date{15.1.2020}
\author{Mihailo Milenković, Dejan Gjer, Bojana Čakarević}

\theoremstyle{definition}
\newtheorem{teorema}{Teorema}[section]
\newtheorem{posledica}{Posledica}[teorema]
\newtheorem{lema}[teorema]{Lema}
\newcommand{\dokaz}[1]{\begin{proof}[Dokaz]#1\end{proof}}
\begin{document}
	
	\maketitle
	
	\newpage
	
	\tableofcontents
	
	\newpage
	
	\section{Uvod}
	
	$ 2^{\frac{k}{2}} \leq R(k,k)$ na osnovu Erdosevog dokaza \cite{theBook}
	
	\lipsum[1]

	\newpage
	
	\section{Primene Remzijeve teoreme}
	%sur i primena sura(citirati Sura)
	\begin{teorema}\label{sur}
		Za svako $k \in \mathbb{N}\textbackslash \{0\}$  postoji neko $n_{0} \in \mathbb{N}$, takvo da za svako bojenje $\chi:\underline{n} \rightarrow \underline{k}$ postoje brojevi $x, y, z \in \underline{n}$ sa osobinom 
		\[
		x + y = z \: \mathrm{i} \: \chi(x)= \chi(y)=\chi(z)
		\]
	\end{teorema}
	\dokaz{
		Neka je $n \in \mathbb{N},\: n+1 \geq R(3)_k=\underbrace{(3,3,\ldots,3)}_\text{k puta}$. Tada ono indukuje sledeće bojenje:
		\[
			\chi^*:[\underline{n+1}]^2\rightarrow \underline{k}:\{i,j\}\mapsto \chi(|i-j|)
		\]
		Zbog $n+1 \rightarrow \underbrace{(3,3,\ldots,3)}_\text{k puta}$, postoje $i_1, i_2$ i $i_3$ obojeni istom bojom, odnosno $\chi^*(\{i_1,i_2\})=\chi^*(\{i_1,i_3\})=\chi^*(\{i_2,i_3\})$. Neka je:
		\[
			x:=i_1-i_2,\:y:=i_2-i_3\:\mathrm{i}\:z:=i_1-i_3
		\]
		Imamo $x,y,z\in \{1,\ldots,n\}$ i $x-y=i_1-i_2+i_2-i_3=i_1-i_3=z$.
	}
	
	\begin{teorema} \label{sur2}
		 Za sve $m \in \mathbb{N} \textbackslash \{0\}$ postoji neko $n_{0} \in \mathbb{N}$, takvo da za sve proste brojeve $p>n_{0}$ jednačina
		\[
		x^{m}+y^{m}\equiv z^{m} (\mathrm{mod} \: p) 
		\]
		ima netrivijalna rešenja. (Rešenje je trvijalno ako $x\cdot y \cdot z\equiv 0\: (\mathrm{mod} \: p)$)
	\end{teorema}
	\dokaz{
		Neka je $n_0=R(3)_m+1$. Neka je $g$ generator grupe $\mathbb{Z}_{p}^{*} $ ($g$ postoji zbog cikličnosti grupe $\mathbb{Z}_{p}^{*}$). Svaki elemenat $x\in\mathbb{Z}_{p}^{*}$ možemo zapisati $x$ kao $g^a$. Imamo $a=mj+i$, za $0\leq i < m$, tako da je $x=g^{mj+i}$. Posmatrajmo bojenje koje boji elemenat $x$ skupa $\mathbb{Z}_{p}^{*}$ u boju $i$ ako je $x=g^{mj+i}$. Na osnovu Šurove teoreme (\ref{sur}), postoje $a, b$ i $c$ obojeni istom bojom, takvi da važi $a+b=c$, odnoso eksponenti $a, b$ i $c$ su kongrueni po modulu $m$. Dakle,
		\[
			g^{mj_{a}+i}+g^{mj_{b}+i}=g^{mj_{c}+i}
		\]
		Neka su $x=g^{j_{a}}$, $y=g^{j_{b}}$ i $z=g^{j_{c}}$. Množenjem gornje jednačine sa $g^{-i}$ dobijamo $x^m+y^m=z^m$  
	}
	%happy ending T.(citirati happyEndingProof rad)
	\begin{teorema}
		Za svaki prirodan broj $n\geq 3$ postoji broj $N(n)$ takav da bilo koji skup od bar $N$ tačaka u ravni u opštem položaju sadrži konveksan $n$-tougao
	\end{teorema}
	\dokaz{
		Za $n=4$ dokazaćemo da $N=5$ zadovoljava uslove. Posmatrajmo 5 tačaka $A,B,C,D \mathrm{i} E$. Ako je najmanji konveksni mnogougao petougao ili četvorougao, dokaz je trivijalan. U suprotnom, neka je najmanji takav mnogougao trougao $ABC$. $D$ i $E$ se onda nalaze unutar $ABC$. 2 tačke od $A$,$B$ i $C$ se moraju nalaziti sa jedne strane prave $DE$. Neka su to $A$ i $C$. Tada je $ACDE$ traženi četvorougao.
		
		Neka je $X$ skup od bar $R_4(n,5)$ tačaka u opštem položaju. Na osnovu Remzijeve teoreme za hipergrafove (\ref{RemziHiper}) znamo da je ovaj broj konačan. Obojimo sve četvoročlane podskupove tačaka u plavo ako je četvorougao koje obrazuju konveksan ili u crveno ako je konkavan. Pošto ima ukupno $R_{4}(n,5)$ tačaka, mora postojati ili $n$-točlani skup tačaka čiji su svi četvoročlani podskupovi plave boje (konveksni) ili petočlani skup tačaka čiji su svi četvoročlani podskupovi crvene boje. Dokazali smo da među 5 tačaka u opštem položaju mora postojati konveksan četvorougao, dakle mora postojati n-točlani skup tačaka tako da su svi četvorouglovi koje oni obrazuju knoveksni, odnosno konveksan n-toguao od n tačaka. Dakle traženi $N$ postoji i važi $N\leq R_{4}(n,5)$
		  
	}
	\newpage
	
\begin{thebibliography}{99}
	%ovde dodati bilo kakve resurse koji se citiraju
	\bibitem{theBook}
	Martin Aigner, Günter M. Ziegler.
	\newblock {Proofs from The BOOK}.
	\newblock Springer, 1998.
	
	\bibitem{mathPaulErdos}
	Ronald L. Graham, Jaroslav Nešetřil, Steve Butler.
	\newblock {The Mathematics of Paul Erdős II}.
	\newblock Springer, 1990.
	
\end{thebibliography}	
	
\end{document}