\documentclass{article}
\usepackage{lipsum}
\usepackage[utf8]{inputenc}
\usepackage[T1]{fontenc}
\usepackage{amsfonts}
\usepackage{amsmath}
\usepackage{amsthm}
\usepackage{hyperref}
\title{Granice za Remzijeve brojeve i primene}
\date{15.1.2020}
\author{Mihailo Milenković, Dejan Gjer, Bojana Čakarević}

\theoremstyle{definition}
\newtheorem{teorema}{Teorema}[section]
\newtheorem{posledica}{Posledica}[teorema]
\newtheorem{lema}[teorema]{Lema}
\begin{document}
	
	\maketitle
	
	\newpage
	
	\tableofcontents
	
	\newpage
	
	\section{Uvod}
	
	$ 2^{\frac{k}{2}} \leq R(k,k)$ na osnovu Erdosevog dokaza \cite{theBook}
	
	\lipsum[1]

	\newpage
	
	\section{Primene Remzijeve teoreme}
	
	\subsection{Šurova teorema}
	\begin{teorema} \label{sur}
		Za svako $k \in \mathbb{N}\textbackslash \{0\}$  postoji neko $n_{0} \in \mathbb{N}$, takvo da za svako bojenje $\chi:\underline{n} \rightarrow \underline{k}$ postoje brojevi $x, y, z \in \underline{n}$ sa osobinom 
		\[
		x + y = z \: \mathrm{i} \: \chi(x)= \chi(y)=\chi(z)
		\]
	\end{teorema}
	
	\begin{teorema} \label{sur2}
		 Za sve $m \in \mathbb{N} \textbackslash \{0\}$ postoji neko $n_{0} \in \mathbb{N}$, takvo da za sve proste brojeve $p>n_{0}$ jednačina
		\[
		x^{m}+y^{m}\equiv z^{m} (\mathrm{mod} \: p) 
		\]
		ima netrivijalna rešenja. (Rešenje je trvijalno ako $x\cdot y \cdot z\equiv 0\: (\mathrm{mod} \: p)$)
	\end{teorema}
	
	
	\newpage
	
\begin{thebibliography}{99}
	%ovde dodati bilo kakve resurse koji se citiraju
	\bibitem{theBook}
	Martin Aigner, Günter M. Ziegler.
	\newblock {Proofs from The BOOK}.
	\newblock Springer, 1998.
	
	\bibitem{mathPaulErdos}
	Ronald L. Graham, Jaroslav Nešetřil, Steve Butler.
	\newblock {The Mathematics of Paul Erdős II}.
	\newblock Springer, 1990.
	
\end{thebibliography}	
	
\end{document}